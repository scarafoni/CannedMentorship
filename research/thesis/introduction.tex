%Introduction
Crowdsourcing has shown to be an effective means of solving tasks that are computationally difficult, or otherwise impractical, for artificial intelligence programs to solve. \cite{howie2006rise}. 
There are many instances, of course,where humans are more effective than machines and vice-versa. 
To pick the most obvious example: humans are more efficient at qualitative, ambiguous tasks, while artificial intelligence (AI) agents tend to be better at more quantitative, direct tasks.
Naturally, most tasks do not fall cleanly into one category or another; there are many tasks which cannot be done perfectly by human or AI agents. 
For this reason, the merging of human and AI is necessary for the completion of a number of tasks.

Consider, for example, the task of building an instruction manual for a common profession. 
If a restaurant wanted to create a manual for a number of everyday tasks in the restaurant in order to train new employees quickly.
It would use crowdsourced data from existing employees to solve this task, as the ambiguous and qualitative data of procedural tasks such as making food and setting tables is far beyond the current capabilities of artificial intelligence.
However, there are many tasks which different employees do differently, and the abi